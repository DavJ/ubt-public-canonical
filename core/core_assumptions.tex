\section*{Core Assumptions (A1–A3) for Emergent $\alpha$ in UBT}

\paragraph{Context: Complex-Time Reduction.}
These core assumptions are formulated in the \textbf{complex-time reduction} of the canonical biquaternion framework as defined in \texttt{canonical/fields/biquaternion\_time.tex} and \texttt{canonical/fields/theta\_field.tex}. The canonical time is the biquaternion time $T_B = t + i\psi + j\chi + k\xi$, and the canonical field is $\Theta(q,T_B) \in \mathcal{B} = \mathbb{H} \otimes \mathbb{C}$. In this document, we work in the isotropic limit where $\chi = \xi = 0$, so that $T_B$ reduces to the complex time $\tau = t + i\psi$, and $\Theta(q,T_B)$ reduces to $\Theta(q,\tau)$.

\paragraph{A1 (BRST Cohomology and Physical Modes).}
Gauge fixing and BRST cohomology project the state space to physical modes; the counting
of effective modes $N_{\mathrm{eff}}$ is invariant under gauge choices and regularization.

\paragraph{A2 (Topological/Spectral Uniqueness of $N_{\mathrm{eff}}$).}
$N_{\mathrm{eff}}$ is uniquely determined by a topological/spectral index associated with
the fundamental domain of the complex-time torus. The value is independent of scheme
parameters $(\mu,\xi)$ and matter parameters (e.g., fermion masses).

\paragraph{A3 (Complex-Time Fibre Periodicity and $R_\psi$).}
The complex-time fibre $\tau=t+i\psi$ is periodic in $\psi$ with quantized radius $R_\psi$
fixed by regularity and single-valuedness of the biquaternion field $\Theta(q,\tau)$.
$R_\psi$ is independent of $(\mu,\xi)$ and matter parameters.

\paragraph{Consequences.}
Under A1–A3, the spatial and temporal invariants
$I_{\mathrm s}=\mathrm{Re}[\Theta^\dagger D_\mu\Theta]_{\mathcal F}$ and
$I_{\mathrm t}=\mathrm{Im}[\Theta^\dagger D_\tau\Theta]_{\mathcal F}$ are finite and scheme/gauge invariant,
and the map $B\leftrightarrow \alpha$ in the Thomson limit is unambiguous.
